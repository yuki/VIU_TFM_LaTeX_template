\newcommand{\ClassPath}{.}
\documentclass{\ClassPath/viu-tfm-template}

% Color principal de la VIU. Si lo modificas, puedes aprovechar
% la plantilla para otras cosas.
\definecolor{maincolor}{HTML}{e65218}

%--------------------------------------------------------------------------
% Definiciones necesarias Modifica con tus datos
%--------------------------------------------------------------------------
\newcommand{\nombre}{Apellido1 Apellido2, Nombre}
\newcommand{\dni}{12345678-A}
\newcommand{\dirige}{Nombre del director/a}
\newcommand{\convocatoria}{Primera}
\newcommand{\titulo}{Título del TFM}
\newcommand{\titulacion}{Máster Universitario en Desarrollo de Aplicaciones y Servicios Web}
\newcommand{\curso}{2022-2023}

\RequirePackage{blindtext} % se puede borrar al escribir el TFM

% importar fichero de Bibliografía
\addbibresource{bibliography.bib}

\begin{document}
    \coverpage

    % opcional en artículos cortos
    \tableofcontents

    \chapter{Ejemplo de cita para bibliografía}
    Tal como aparece en \textcite{einstein}, ...

    \Blinddocument % ejemplo de texto, del paquete “blindtext”

    % sólo si se tiene bibliografía
    \printbibliography[title={Referencias bibliográficas},heading=bibintoc]
\end{document}